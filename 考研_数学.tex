%% LyX 2.2.3 created this file.  For more info, see http://www.lyx.org/.
%% Do not edit unless you really know what you are doing.
\documentclass[UTF8]{ctexart}
\usepackage{amsmath}
\usepackage{amssymb}
\usepackage{fontspec}
\usepackage[unicode=true,pdfusetitle,
 bookmarks=true,bookmarksnumbered=true,bookmarksopen=true,bookmarksopenlevel=1,
 breaklinks=false,pdfborder={0 0 1},backref=false,colorlinks=false]
 {hyperref}

\makeatletter
%%%%%%%%%%%%%%%%%%%%%%%%%%%%%% User specified LaTeX commands.
% 重定义\nobreakspace命令,避免XeTeX出错(2.1.2版本后已不需要)
\DeclareRobustCommand\nobreakspace{\leavevmode\nobreak\ }

% 解决pdflatex不支持插入eps图片问题
\usepackage{epstopdf}
\usepackage[left=2cm, right=2cm, top=2cm]{geometry}

\makeatother

\usepackage{xunicode}
\begin{document}
1.当题目中有关于积分的不等式时,可以考虑构造一个原函数$F(x)$使$F^{\prime}(x)$有不等式的形式,如$F(x)=x\int_{x}^{1}f(y)dy$或直接求它的导函数$f^{\prime}(x)$;使用积分中值定理{[}与积分面积相等的矩形必与曲线相交于一点$(\xi,f(\xi))${]}

2.当题目中有等式或函数某点值时,可考虑拉格朗日中值定理和罗尔定理.(拉格朗日中值定理与一阶泰勒公式等价.)

3.题目中有定积分时可考虑将常数项凑成定积分.

4.注意定义域.

5.$(\ln(x+\sqrt{x^{2}\pm1}))^{\prime}=\frac{1}{\sqrt{x^{2}\pm1}}$,$(\arcsin x)^{\prime}=\frac{1}{\sqrt{1-x^{2}}}$,$(\frac{1}{2}x\sqrt{x^{2}\pm1}\pm\frac{1}{2}\ln|x+\sqrt{x^{2}\pm1}|)^{\prime}=\sqrt{x^{2}\pm1}$,$(\frac{1}{2}x\sqrt{1-x^{2}}+\frac{1}{2}\arcsin x)^{\prime}=\sqrt{1-x^{2}}$(可用dx中的x分部积分得出)

6.分部积分时设积分结果为$I$,要尽量凑出$I$以构成关于$I$的方程,如$I=\int_{0}^{\frac{m\pi}{2}}\sin^{n}xdx$要尽量使右侧出现$\sin x$.

7.能用三角解决的不要用无理式,有$\sin^{2}x$,$\cos^{2}x$可凑$\tan^{2}x$,有$\frac{1}{\cos^{2}x}dx$可凑$d(\tan x)$,有$\sqrt{1-x^{2}}$可换$\sin t$,想办法凑$\tan t$.

8.$\int\frac{du}{(\frac{u}{2})^{2}+1}\neq\arctan\frac{u}{2}$不要忘记积分变量与变量一致性.

9.积分结果一定要求导验证.

10.能拆出常数项或多项式项的要先拆后积分,如$\int\frac{du}{(2u-1)(u^{2}+1)}=\int\frac{u^{2}+1-u^{2}}{(2u-1)(u^{2}+1)}du$从高阶开始配凑,也可使用待定参数,令$\frac{1}{(2u-1)(u^{2}+1)}=A\frac{1}{2u-1}+B\frac{u}{u^{2}+1}+C\frac{1}{u^{2}+1}$解得$A=\frac{4}{5},B=-\frac{2}{5},C=-\frac{1}{5}$.归纳:分母要化为$(A_{1}x+B_{1})^{n}(A_{2}x^{2}+B_{2})(A_{3}x^{2}+B_{3}x+C_{3})$时,若后两式不能因式分解则最简分量必有$\frac{1}{(A_{1}x+B_{1})^{n}},\frac{x}{A_{2}x^{2}+B_{2}},$$\frac{1}{A_{2}x^{2}+B_{2}},$

$\frac{x}{A_{3}x^{2}+B_{3}x+C_{3}},\frac{1}{A_{3}x^{2}+B_{3}x+C_{3}}$.

11.柯西不等式$[\int_{a}^{b}f(x)g(x)dx]^{2}\leq\int_{a}^{b}[f(x)]^{2}dx\cdot\int_{a}^{b}[g(x)]^{2}dx$.

12.直线$l$$\frac{x-x_{0}}{a}=\frac{y-y_{0}}{b}=\frac{z-z_{0}}{c}$方向向量为$\vec{l}=(a,b,c)$,经过点$L(x_{0},y_{0},z_{0})$,点$M$到$l$的距离为$d=\frac{|\vec{l}\times\vec{L}|}{|\vec{l}|}$.

13.一条直线与另外两条直线相交,它们的方向向量没有直接关系,也不能联立两直线求交点,认为它在第三条直线上,因为交点可能有两个.

14.若$f(x,y)$在$(x,y)$点可微,则$f(x+\Delta x,y+\Delta y)-f(x,y)=f'_{x}(x,y)\Delta x+f'_{y}(x,y)\Delta y=O(\sqrt{(\Delta x)^{2}+(\Delta y)^{2}})$.

15.$\frac{d}{dy}f(0,y)|_{y=0}=\lim\limits_{\Delta y\to0}\frac{f(0,\Delta y)-f(0,0)}{\Delta y}=\lim\limits_{\Delta y\to0}[\frac{f(x,\Delta y)-f(x,0)}{\Delta y}]_{x=0}=\frac{\partial}{\partial y}f(x,y)|_{(0,0)}$,因此可在求偏导前对无关变量赋值.

16.设$\vec{l}$与$x,y,z$轴夹角分别为$\alpha,\beta,\gamma$,则函数$u(x,y,z)$沿$\vec{l}$方向的方向导数为$\frac{\partial u}{\partial\vec{l}}=\frac{\partial u}{\partial x}\cos\alpha+\frac{\partial u}{\partial y}\cos\beta+\frac{\partial u}{\partial z}\cos\gamma$.

17.$[f(u(x,y),v(x,y))]_{x}^{\prime}=f_{1}^{\prime}u_{x}^{\prime}+f_{2}^{\prime}v_{x}^{\prime}.$

18.$u(x,y,z)$在$\nabla u$方向上的方向导数最大,值为$|\nabla u|$.

19.$f_{x}^{\prime}(x,y,z)$ $(z=z(x,y))$ $\neq$ $\frac{\partial}{\partial x}[f(x,y,z(x,y))]$.

20.$\varphi_{x}^{\prime}(2x^{2})=\varphi^{\prime}(2x^{2})\cdot4x$.

21.当$f_{12}^{\prime\prime}=f_{21}^{\prime\prime}$时,最后结果应合并.

22.隐函数求导之前应将所有$z=z(x,y)$标记出来以免遗忘.

23.多项式要统一格式,如$x-y,y-x$应统一为$x-y,-(x-y)$.

24.求导要一步一步地求,不要跳步.

25.对$f(x,y)=0$可用全微分为0得到$\frac{dy}{dx}=-\frac{f_{x}^{\prime}}{f_{y}^{\prime}}$.

26.$\frac{\partial(x,y)}{\partial(u,v)}=\begin{vmatrix}\frac{\partial x}{\partial u} & \frac{\partial x}{\partial v}\\
\frac{\partial y}{\partial u} & \frac{\partial y}{\partial v}
\end{vmatrix}=\begin{vmatrix}\begin{bmatrix}\frac{\partial x}{}\\
\frac{\partial y}{}
\end{bmatrix}\begin{bmatrix}\frac{}{\partial u} & \frac{}{\partial v}\end{bmatrix}\end{vmatrix}$.

27.曲线$(x(t),y(t),z(t))$在$(x,y,z)$点切线方向向量为$(x^{\prime},y^{\prime},z^{\prime})$.

28.曲面$f(x,y,z)=0$在$(x,y,z)$点法线方向向量为$(\frac{\partial f}{\partial x},\frac{\partial f}{\partial y},\frac{\partial f}{\partial z})$.

29.$\vec{a}=(x_{1},y_{1},z_{1})$,$\vec{b}=(x_{2},y_{2},z_{2})$方向相同$\Rightarrow$$\frac{x_{1}}{x_{2}}=\frac{y_{1}}{y_{2}}=\frac{z_{1}}{z_{2}}$$\nRightarrow$$\begin{cases}
x_{1}=x_{2}\\
y_{1}=y_{2}\\
z_{1}=z_{2}
\end{cases}$.

30.求函数在曲面上的最值(条件最值),因最值点一定在曲面上,可将曲面作为乘子引入,令$F=(\ln)f(x,y,z)+\lambda\cdot g(x,y,z)$(拉格朗日乘子法).

31.海伦公式:$S=\sqrt{\frac{L}{2}(\frac{L}{2}-a)(\frac{L}{2}-b)(\frac{L}{2}-c)}$.

32.$z(x,y)$在$(x_{0},y_{0})$处满足$\frac{\partial z}{\partial x}=\frac{\partial z}{\partial y}=0$,$A=\frac{\partial^{2}z}{\partial x^{2}},B=\frac{\partial^{2}z}{\partial x\partial y},C=\frac{\partial^{2}z}{\partial y^{2}}$,若:1.$B^{2}-AC<0$,
$1^{\circ}$.$A>0$,取极小值,$2^{\circ}$.$A<0$,取极大值;2.$B^{2}-AC>0$,非极值点.

33.对$f(x,y)$若$\frac{\partial^{2}f}{\partial x\partial y}=0$,则$\frac{\partial f}{\partial y}=\int0dx=\varphi(y)$,$f=\int\varphi(y)dy=\psi(y)+C(x)$,对变量$x$积分会出现$\varphi(y)$.

34.不要把法向量与$(\frac{\partial F}{\partial x},\frac{\partial F}{\partial y},\frac{\partial F}{\partial z})$划等号,它们是共线关系.

35.多重积分时,$x$的上下限为区域画横线,$y$的上下限为区域画竖线.

36. 曲面面积$S=\iint\sqrt{1+(z_x^{\prime})^2+(z_y^{\prime})^2}\mathrm{d}x\mathrm{d}y$

37. $(x^2)^{\frac{3}{2}}\neq x^3$

38. 函数奇偶性,周期性结合积分上下限可极大简化运算

39. 多重积分$ \int \mathrm{d}x \int \mathrm{d}y \int f \mathrm{d}z$中y的上下限由对z积分后降维图形区域范围确定

40. 转动惯量$ I=\iiint r^2\rho\mathrm{d}V$

41. 开根号一定要加绝对值

42. 不要轻易破坏对称性用于化简,如$ \int_{-\frac{\pi}{2}}^{\frac{\pi}{2}}\sin\theta\cos^3\theta\mathrm{d}\theta$因$\sin\theta\cos^3\theta$为奇,得积分为0

43. $\int_C P\mathrm{d}x+Q\mathrm{d}y$与路径无关$\Leftrightarrow\oint_CP\mathrm{d}x+Q\mathrm{d}y =0\Leftrightarrow P\mathrm{d}x+Q\mathrm{d}y=\mathrm{d}u\Leftrightarrow\frac{\partial Q}{\partial x}=\frac{\partial P}{\partial y}$,设$\vec{F}=(P(x,y,z),Q(x,y,z),R(x,y,z)),\\ \mathrm{d}\vec{l}=(\mathrm{d}x,\mathrm{d}y,\mathrm{d}z),\\ \oint_L\vec{F}\mathrm{d}\vec{l}=\oint_LP\mathrm{d}x+Q\mathrm{d}y +R\mathrm{d}z =\iint_\Sigma\nabla \times\vec{F}\cdot\mathrm{d}\vec{\sigma}$对于可分离出保守场的积分要进行分离以简化运算

44. 设平面$\Sigma(x,y,z)$法向量为$\vec{p}=(A,B,C)$面积为S,则它在$\vec{n}=(\alpha,\beta,\gamma)$方向降维后面积为$S^{\prime}=S\frac{\vec{n}\cdot\vec{p}}{|\vec{n}|\cdot|\vec{p}|}$,其中$\vec{p}$的方向与围成$\Sigma$的封闭曲线有关,从z轴正向看即从z的正无穷处向下看,逆时针则$\vec{p}$向上

45. 求体积分和平面积分时,不要用边界条件化简被积函数

46. 无穷级数若可展开为$\frac{1}{n}+\frac{1}{n^2}+\cdots$则应展开后根据最低阶判断收敛性(比较判别法的极限形式)没有$\frac{1}{n}$可先凑出$\frac{1}{n}$;可积分时使用积分判别法;带有$p^n$时考虑$\lim\limits_{n\to\infty}\sqrt[n]{a_n}$判别(根值判别法)

47. 使用洛必达后再用级数展开会引起系数错误,如$1-\frac{\sin x}{x}$

48. 交错级数:
    1. 求$\lim\limits_{n\to\infty}|a_n|$,不为0则不收敛,再看是否绝对收敛
    2. 求$a_n$与$a_{n+1}$大小关系,若$a_n>a_{n+1}$,条件收敛

49. $\sum\limits_{n=1}^{+\infty}\frac{1}{n}=\ln n+c+r_n$,c为欧拉常数,$r_n\to0,\sum\limits_{n=1}^{\infty}\frac{1}{n^2}=\frac{\pi^2}{6}$

50. $\because \frac{\ln n}{n}>\frac{1}{n}, (n\geq3), \therefore\sum\limits_{n=1}^{+\infty}\frac{\ln n}{n}$发散

51. 一阶线性微分方程$\frac{\mathrm{d}y}{\mathrm{d}x}+P(x)y=Q(x)$先解齐次方程通解$y=C_1e^{-\int P(x)\mathrm{d}x}$令$C_1=C_1(x)$将通解代入非齐次方程得$y=e^{-\int P\mathrm{d}x}[c+\int Q\mathrm{e}^{\int P\mathrm{d}x}\mathrm{d}x]$

52. 复杂级数先求导或积分再求幂级数

53. $b_n=\frac{2}{T}\int_0^Tf(x)\sin\frac{n\pi x}{T}\mathrm{d}x$ (延拓), $b_n=\frac{2}{2T}\int_{-T}^Tf(x)\sin\frac{n\pi x}{T}\mathrm{d}x$ 

54. 和差化积
    $$
    \sin\alpha+\sin\beta=2\sin\frac{\alpha+\beta}{2}\cos\frac{\alpha-\beta}{2}\\
    \cos\alpha+\cos\beta=2\cos\frac{\alpha+\beta}{2}\cos\frac{\alpha-\beta}{2}
    $$

55. 解一阶非线性方程时,将x和y凑到一起再进行换元,如$(x+y)y^{\prime}+(x-y)=0$ ,令$u=\frac{y}{x},y^\prime=\frac{u-1}{u+1}$

56. 定积分不要忘记减下限

57. $P(y,x)\mathrm{d}x+Q(y,x)\mathrm{d}y=0$形式的方程:先看$\frac{\partial P}{\partial y}=\frac{\partial Q}{\partial x}$是否成立,成立则用线积分,或考虑积分因子;再看是否可化为$y^\prime+Py=Q$形式,可以就用公式;最后看是否可分离变量,换元

58. 曲率$\kappa={|\frac{\mathrm{d}\alpha}{\mathrm{d}s}|}=\frac{\mathrm{d}(\arctan y^\prime)}{\sqrt{1+y^{\prime2}}\mathrm{d}x}=\frac{|y^{\prime \prime}|}{(1+y^{\prime2})^{\frac{3}{2}}}$ 

59. 方程$y^{\prime \prime}+Py^\prime+qy=P_m(x)\mathrm{e}^{ax}$,特解为$y^*=x^kQ_m(x)\mathrm{e}^{ax}$,k为$\mathrm{e}^{ax}$根重数,m为阶数

60. 方程$y^{\prime \prime}+Py^\prime+qy=P_m(x)\mathrm{e}^{ax}\sin bx$,特解为$y^*=x^k(R_m(x)\cos bx + S_m(x)\sin bx)\mathrm{e}^{ax}$,k与a无关,k为$a+ib$的根重数

61. 解伯努利方程$y^\prime+P(x)y=q(x)y^n$应令$z=y^{1-n}$ ;解欧拉方程$x^2\frac{\mathrm{d}^2y}{\mathrm{d}x^2}+a_1x\frac{\mathrm{d}y}{\mathrm{d}x}+a_2y=f(x)$ ,令$x=\mathrm{e}^t$

62. 逆序数:从左到右每个数右边小于它的个数之和

63. $AX=B$有解$\Leftrightarrow \mathrm{rank}(A)=\mathrm{rank}(\bar{A})$ 其中$\bar{A}=[A|B]$为增广矩阵

64. 左/右乘逆矩阵$\Leftrightarrow$增广行/列变换;行/列增广时,逆矩阵的原矩阵的放在左边/上边

65. 行列式为0$\Leftrightarrow$齐次线性方程组有非0解

66. 齐次线性方程组可写成列向量线性组合和行向量点乘形式

67. 实对称阵,不同特征值对应的不同特征向量正交

68. 实对称$\Rightarrow$相似于对角阵;n个不同特征值$\Rightarrow$ 相似对角;n个特征向量线性无关$\Leftrightarrow$ 相似对角

69. $\sum\limits_{i=1}^n\lambda_i=\sum\limits_{i=1}^na_{ii}=tri(A),\prod\limits_{i=1}^n\lambda_i=|A|$  ,相似$\Rightarrow$同秩,同特征值,同行列式

70. 设特征向量$\alpha_1,\alpha_2\cdots\alpha_n$对应特征值$\lambda_1,\lambda_2\cdots\lambda_n$ ,则$[\alpha_1\cdots\alpha_n]^{-1}A[\alpha_1\cdots\alpha_n]=\left[\begin{matrix}\lambda_1&\cdots&0\\0&\cdots&0\\\vdots&&\vdots\\0&\cdots&\lambda_n\end{matrix}\right]=\Lambda$

71. 一个特征值至少有一个特征向量

72. $A=\left[\begin{matrix}a&b\\c&d\end{matrix}\right],A^*=\left[\begin{matrix}d&-b\\-c&a\end{matrix}\right],A^{-1}=\frac{1}{ad-bc}\left[\begin{matrix}d&-b\\-c&a\end{matrix}\right]$

73. 标准型可与相似对角阵无关,但满足惯性定律

74. 随机事件常用技巧:加一个有的,$(A+B)+B=C+B$;乘一个有的,$(AB)B=CB$;拆成独立事件

75. $C_0^0$不存在,当$P(B)=0$时,$P(A|B)$不存在

76. $f(x)=\frac{1}{\sqrt{2\pi}\sigma}\mathrm{e}^{-\frac{(x-\mu)^2}{2\sigma^2}}$

77. 切比雪夫大数定律:$P\left\{\text{偏差比}\epsilon\text{大}\right\}\leq\frac{\text{方差}}{\epsilon^2}$;依概率收敛$n\rightarrow+\infty$时,偏差小于任意小常数;切比雪夫大数定律:$n\rightarrow+\infty$时,总偏差的平均值小于任意小常数;伯努利大数定律:为二项分布时大数定律的特例;辛钦大数定律:二项分布且$E(x_i)=\mu$时大数定律的特例;棣莫弗-拉普拉斯中心极限定律:二项分布当n很大时,$\frac{\text{偏差}}{\text{标准差}}$服从正态分布;列维-林德伯格中心极限定理:$\sqrt{n}\frac{\text{总偏差的平均值}}{\text{标准差}}$服从正态分布

78. $D(X-Y)=D(X)+D(Y)-2\mathrm{Cov}(x,y)$

79. $\sigma$和$\sigma^2$ 注意区分,指数分布$P(x)=\lambda\mathrm{e}^{-\lambda x},E(X)=\frac{1}{\lambda},D(X)=\frac{1}{\lambda^2}$泊松分布$P(X=k)=\frac{\lambda^k}{k!}\mathrm{e}^{-\lambda},E(X)=D(X)=\lambda$

80. $X+Y\sim N(2\mu,2\sigma^2),X-Y\sim N(0,2\sigma^2)$,X,Y满足独立,正态分布

81. $\int_0^{+\infty}x^n\mathrm{e}^{-x}\mathrm{d}x=n!,\int_0^{+\infty}\frac{x^n}{n!}\mathrm{e}^{-x}\mathrm{d}x=1$

82. 正态分布运算法则:$D(X_1+X_2+X_3)=3\sigma^2,D(X_1+2X_2)=\sigma^2+4\sigma^2$

83. 正态总体的抽样分布中,$\bar{X}$与$S^2$相互独立($\mu=0,\sigma=1$)$n\bar{X}^2\sim\chi^2(1),(n-1)S^2\sim\chi^2(n),E(\chi^2(n))=n,D(\chi^2(n))=2n$

84. 若$\alpha$为单位向量,即$\alpha^T\alpha=1$,则$\alpha(\alpha^T\cdot\alpha)=\alpha\cdot1=1\cdot\alpha$,即1为$\alpha\alpha^T$的唯一非零特征值

85. 矩阵与对角阵相似时,对角阵中特征值顺序不影响。矩阵A特征值$\lambda_1\cdots\lambda_n$,则$A\sim\left[\begin{matrix}\lambda_1&\cdots&0\\0&\cdots&0\\\vdots&&\vdots\\0&\cdots&\lambda_n\end{matrix}\right]\Leftrightarrow$有n个线性无关特征向量$\Leftrightarrow\lambda_1=\cdots=\lambda_n(=\lambda)$时,$A-\lambda E$秩为n-r(r重根)

86. 若$X\sim\chi^2(n)$,则2X不服从$\chi^2$分布,因此先化系数

87. 绕y轴旋转的绕点会构成一个以y轴为圆心的圆,即$\sqrt{x^2+z^2}=r$,y坐标不变,因此$\frac{x^2}{a^2}-\frac{y^2}{b^2}=1\Rightarrow\frac{x^2+z^2}{a^2}-\frac{y^2}{b^2}=1$

88. 非实对称阵无对应的正交对角化阵,即不存在$P^TP=E$,使$P^{-1}AP=\Lambda$(其中A为任一非对称阵)

89. $\int_0^{+\infty}\frac{f(x)}{x^p}\mathrm{d}x=\lim\limits_{\epsilon\to0}\int_\epsilon^{1}\frac{f(x)}{x^p}\mathrm{d}x+\lim\limits_{N\to+\infty}\int_1^{N}\frac{f(x)}{x^p}\mathrm{d}x$可判断敛散性,利用极限由洛必达法则还可求出等价积分式;当p<1时,$\int_0^1\frac{1}{x^p}\mathrm{d}x$存在,当p>1时,$\int_1^{+\infty}\frac{1}{x^p}\mathrm{d}x$存在

90. 前n-1阶导数为0,第n阶不为0$\begin{cases}\text{若n为奇,拐点}\\\text{若n为偶,极值点}\end{cases}$

91. 拐点只能在二阶导数为0或不存在的点取到;极值点只能在驻点或导数不存在的点取到

92. $f(x,y)$在$(x,y)$点最大方向导数为$|\vec{\nabla}f|$(模)

93. 计算正态分布概率大小要化成标准正态分布再比较

94. 平面图形或线图形绕轴旋转应找到三维坐标与平面某长度关系

95. 看到$a_{ij}$先想到转化成矩阵

96. 代数运算时应把x写在一起,把y写在一起减少失误

97. 出现$f(x)+f^\prime(x)$时考虑构造$g(x)=f(x)\mathrm{e}^x$

98. 解非齐次线性方程组时不要直接找线性无关解向量,应先转化为齐次线性方程组找到线性无关解再结合特解得到通解

99. $S=\frac{1}{x}\sum\limits_{n=0}^{\infty}\frac{x^{2n+1}}{2n+1}\neq\int\frac{1}{x}\sum\limits_{n=0}^{\infty}x^{2n}\mathrm{d}x$提出的x在求导时会影响结果,应令$S=\frac{1}{x}f(x),f^{\prime}(x)=\sum\limits_{n=0}^{\infty}x^{2n}$

100. $+\infty$和$-\infty$可能对应两条水平渐近线或一条,奇点处常有一条垂直渐近线,有两条水平渐近线就不可能有斜渐近线

101. $f(x,y)$在$(x_0,y_0)$可微$\Leftrightarrow\Delta z=z_x^\prime\Delta x+z_y^\prime \Delta y+O(P)\Leftarrow f^\prime_x, f^\prime_y连续$

102. 读题要仔细,不要把$\mathrm{e}^{x^2}$看成$\mathrm{e}^{-x^2}$

103. 分清收敛区间(开区间)和收敛域

104. 单调有界$\Rightarrow$收敛

\end{document}
